%+++++++++++++++++++++++++++++++++++++++++++++++++++++++++++++++++++++
%Informações do projeto de TCC:
\titulo{ITA HELP: SITE PARA OTIMIZAÇÃO DO COMÉRCIO DE SMARTPHONES}
\newcommand{\tituloingles}{ITA HELP: WEBSITE FOR SMARTPHONE TRADE OPTIMIZATION}
\autor{Douglas Tavares de Oliveira Alves Ramos}
\orientador{Prof. Dr. Ramiro Tadeu Wisnieski}
\coorientador{} %Deixe em branco caso não haja coorientador
\data{\the\year{}} % Apenas o ano
\local{Itapetininga} % Apenas a cidade

% Insira aqui até cinco palavras chave separadas por ponto. Faça o mesmo para as palavras em inglês. As palavras chave serão automaticamente inseridas no resumo e abstract. O resumo e abstract devem ser aditados no arquivo resumo.tex
\newcommand{\palavraschave}{Palavra 1; Palavra 2; Palavra 3.}
\newcommand{\keywords}{Palavra 1; Palavra 2; Palavra 3.}

% Informações da Banca:

% Entre com os dados dos Membros da banca

% Instituição do(a) Orientador(a)
\newcommand{\oriInst}{IFSP - Campus Itapetininga}

% Primeiro membro:
%\newcommand{\membroA}{Prof. Josef Climber} % Nome completo
%\newcommand{\membroAinst}{UTFPR-TD} % Instituição

% Segundo Membro:
%\newcommand{\membroB}{Prof. Jhonny Epaminomdas} % Nome completo
%\newcommand{\membroBinst}{Unicamp} % Instituição

% Data de defesa:
% \newcommand{\Data}{} 

