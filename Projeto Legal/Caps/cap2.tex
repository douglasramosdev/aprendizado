\chapter{Justificativa}

O mercado de smartphones usados tem registrado um crescimento expressivo
nos últimos anos, impulsionado pelo aumento dos preços de dispositivos novos e pela
busca por alternativas mais acessíveis e sustentáveis.

Há uma previsão de crescimento na comercialização de smartphones usados mundialmente, projetados entre os anos
de 2022 à 2032, a empresa Custom Market Insights, em conjunto de outras empresas
desenhou um gráfico CAGR (taxa de crescimento anual composta) para ilustrar esse crescimento
anual (em Dólares) na comercialização desses aparelhos:

\begin{figure}[!htb]
    \centering
    \caption{Taxa de crescimento anual composta de Smartphones Usados(CAGR)}
    \includegraphics[width=0.7\textwidth]{Caps/Figs/Figura 1 - CAGR dos Anos de 2022 à 2032.jpg}
    \fonte{Custom Market Insights, 2022}
    \label{fig:figura-exemplo1}
\end{figure}
\begin{citacao}
    O uso de celulares usados é cada dia mais frequente por diversos motivos; aqueles que buscam um preço baixo, ou quem quer preservar o meio ambiente, ou mesmo aqueles que não querem dar um aparelho tão caro para seus filhos.
    
    O fato é que o mercado de celulares usados cresceu nos últimos dois anos. De acordo com dados da pesquisa Panorama Mobile Time / Opinion Box, de julho de 2022, aparelhos usados correspondem a 9\% das vendas de smartphones no Brasil, sendo a maior parte deles adquirida por jovens de 16 a 29 anos (13\%) e entre classes C, D e E (10\%) — nas classes A e B, a porcentagem é 4\%.
    
    Para fins de comparação, em 2020 o número de usados vendidos correspondia a 7,2\% do total, segundo a IDC Brasil — que aponta que esse mercado está crescendo.
    \end{citacao}
    
    \citeonline{fontes2022}  

    \break
    Além do fator comercial e econômico, esse segmento impacta também no meio ambiente, como indicam os dados obtidos pelos autores à seguir:
    
    \begin{citacao}
        A escassez no fornecimento de matéria-prima, que interrompeu a produção de automóveis durante todo o ano passado e este ano, está cada vez mais próxima dos fabricantes de eletrônicos, equipamentos médicos e outros produtos industriais.
        
        Esses últimos estão inclusive adaptando a produção e revendo o cronograma por conta do aumento dos preços e da falta dos chips semicondutores.
        
        Atualmente, no Brasil, quatro em cada dez fábricas de eletrônicos, como TVs, notebooks e celulares, já tiveram que paralisar as atividades ou sofreram atrasos na produção. Até a gigante Apple já anunciou que espera que os problemas de fornecimento se espalhem para os iPhones e iPads.
        
        "A falta dos semicondutores na indústria automotiva é só a ponta do iceberg. Uma máquina em hospital de ressonância, os computadores avançados para certos exames, em algum momento, vão começar a faltar também”, analisa John Paul Lima, coordenador do curso de engenharia da Faculdade de Informática e Administração Paulista (Fiap).
        
        "Segundo um levantamento realizado pela Associação Brasileira da Indústria Elétrica e Eletrônica (Abinee), 12\% dos fabricantes do setor tiveram que parar parte da produção no mês passado por falta de componentes eletrônicos. Esse é o maior registro desde que, em fevereiro, a pesquisa começou a acompanhar o impacto da falta de insumos no mercado brasileiro.
        
        Ainda de acordo com a pesquisa, 32\% das empresas relataram enfrentar atrasos na produção ou na entrega de produtos aos clientes.
        
        Entre os fabricantes de produtos que contêm semicondutores, 71\% das companhias afirmaram ter tido dificuldades para adquirir o insumo. No mês anterior, a proporção estava em 55\%.
        
        Por conta desse problema, 93\% das fábricas sentiram o impacto no preço dos componentes. As principais dificuldades citadas são a escassez de chips (80\%), o aumento das tarifas de frete (71\%) e a desvalorização cambial (44\%).
        \end{citacao}
        
        \citeonline{fontes2021}

        \break 

        \begin{citacao}
            Além disso, trouxe benefícios para a população de partes interessadas. Em relação aos \textbf{legisladores}, este método tem características ecológicas, o que leva à redução de resíduos desnecessários. Para as empresas, essa estratégia pode moderar o dano, reduzindo as perdas com sucata. Quando os itens devolvidos são reformados, algumas vantagens são oferecidas aos consumidores, pois diferentes opções são apresentadas a eles, além da oportunidade de economia (Yoo e Kim, 2016).
            
            Os avanços em Tecnologia da Informação (TI) aumentaram a demanda por equipamentos eletrônicos (Deng et al., 2017). Nossa pesquisa focou nos smartphones como um produto de ciclo de vida curto na indústria de Tecnologia da Informação (TI). O número de usuários de smartphones no mundo todo deve ter atingido 3,2 bilhões em 2019, e estima-se que aumente para 3,8 bilhões até o final de 2021 (Statista, 2019a). Esse aumento rápido no número de usuários e na popularidade do produto levou ao aumento da demanda por matérias-primas na fabricação. Simultaneamente, a inovação tecnológica e a expansão do mercado criam pressão para atualizar a tecnologia nos smartphones, substituindo modelos pela versão mais recente, o que encurta a vida útil do produto. De acordo com a Statista (2019b), a vida útil média dos smartphones no mundo em 2020 foi estimada em 2,8 anos. Isso significa que o ciclo de vida curto dos smartphones e os cenários limitados de fim de vida (EOL) levam ao lixo eletrônico (e-waste) e à perda de materiais escassos.
            
        \end{citacao}
            
        \citeonline{nasiri2020}


        Vimos um crescimento expressivo econômico e comercial nos ultimos 5 anos pós-pandemia,
        acarretado pelas crises, tanto com a pausa nas produções industriais, quanto a escassez de 
        matéria-prima para a fabricação de novos aparelhos, gerando assim, um aumento da demanda por 
        aparelhos novos.
        Essas crises vieram de sanções econômicas à países envolvidos em conflitos armados, e também pela
        falta de mão de obra nas industrias devido ao fechamento das fábricas durante a Pandemia do COVID-19 em 2020.
        Os dados apresentados, justificam a elaboração do nosso projeto, que pretende facilitar e movimentar este comércio
        em ascenção.
        

\label{chap:Justificativa}