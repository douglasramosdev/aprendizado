\chapter{Introdução}
    A constante evolução da tecnologia e o crescimento exponencial do mercado
de dispositivos móveis têm impulsionado a busca por alternativas sustentáveis e
econômicas no consumo de smartphones. 

Neste contexto, o presente trabalho propõe
o desenvolvimento de um site que permita a troca de smartphones usados por novos,
criando uma plataforma que incentiva a economia circular e reduz o impacto ambiental
associado ao descarte de resíduos eletrônicos.
O projeto visa unir praticidade e inovação tecnológica, demonstrando a
aplicação de conceitos de desenvolvimento web, engenharia de software e
o uso de metodologias ágeis para criar uma solução funcional e eficiente.
Este projeto está sendo realizado a pedido da empresa Ita Help Assistência Técnica
especializada em celulares, localizada na cidade de Itapetininga - SP. 

Além de atender a uma necessidade real de mercado, este trabalho serve como uma
oportunidade para consolidar conhecimentos adquiridos ao longo do curso,
contribuindo para o amadurecimento acadêmico e profissional.
A abordagem considera aspectos técnicos, como a utilização de tecnologias
modernas, e organizacionais, como planejamento detalhado e gestão de recursos,
para garantir que o sistema atenda aos requisitos propostos.

Foram elaborados escopo e análise de requisitos em cima do que foi proposto pela empresa
para a implementação do website, incluindo funcionalidades como gerenciamento de inventário de smartphones,
e um sistema de avaliação para facilitar as trocas. O desenvolvimento será realizado 
utilizando tecnologias como HTML, CSS, JavaScript e frameworks modernos, garantindo uma 
interface amigável e responsiva.
\label{chap:intro}